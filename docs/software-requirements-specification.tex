\documentclass[a4paper]{article}
\usepackage[utf8]{inputenc}
\usepackage{float}
\usepackage{caption}

\captionsetup[table]{labelformat=empty}
\newcommand{\bugTracker}{\textbf{BugTracker}}

\begin{document}

    \begin{titlepage}
        \vspace*{\fill}
        \begin{center}
            \huge
            BugTracker

            \vspace{0.4cm}
            \Huge
            Software Requirement Specification

            \vspace{3cm}
            \begin{table}[h]
                \centering
                \caption{Document Revision History}
                \begin{tabular}{cccc}
                    \hline
                    Date & Changed By & Version & Comments \\
                    \hline
                    2022-03-13 & Artur Kristof & 0.1 & Minimum viable product \\
                    \hfill & \hfill & \hfill & \hfill \\
                    \hline
                \end{tabular}
            \end{table}
        \end{center}
        \vspace*{\fill}
    \end{titlepage}

    \tableofcontents

    \section{Introduction}
    This document details the project plan for the developments of \bugTracker{}.
    \subsection{Purpose}
    The purpose of this project is to create a web application that can be used as a backlog of tickets, bugs, tasks, user stories etc. by development teams working on building software.
    Out of all the documents available in \bugTracker{} documentation this one is of highest abstraction, which means that it only defines \textit{what} features will be available
    and not \textit{how} they will look or \textit{how} they will be implemented.
    \subsection{Intended Audience}
    \subsection{Scope}
    \section{Overall Description}
    \subsection{User Needs}
    \subsection{Assumptions and Dependencies}
    \section{System Features and Requirement}
    \subsection{Functional Requirements}
    \subsection{External Interface Requirements}
    \subsection{System Features}
    \subsection{Non-functional Requirements}

\end{document}