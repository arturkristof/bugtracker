\documentclass[a4paper]{article}
\usepackage[utf8]{inputenc}
\usepackage{float}
\usepackage{caption}

\captionsetup[table]{labelformat=empty}
\newcommand{\appName}{BugTracker}
\newcommand{\appNameBold}{\textbf{BugTracker}}

\begin{document}

    \begin{titlepage}
        \vspace*{\fill}
        \begin{center}
            \huge
            \appName{}

            \vspace{0.4cm}
            \Huge
            Software Requirement Specification

            \vspace{3cm}
            \begin{table}[h]
                \centering
                \caption{Document Revision History}
                \begin{tabular}{cccc}
                    \hline
                    Date & Changed By & Version & Comments \\
                    \hline
                    2022-03-13 & Artur Kristof & 0.1 & Minimum viable product \\
                    \hfill & \hfill & \hfill & \hfill \\
                    \hline
                \end{tabular}
            \end{table}
        \end{center}
        \vspace*{\fill}
    \end{titlepage}

    \tableofcontents

    \section{Introduction}
    This document details the project plan for the developments of \appNameBold{}.

    \subsection{Purpose}
    The purpose of this project is to create a web application that can be used as a backlog of tickets, bugs, tasks, user stories etc. by development teams working on building software.
    Out of all the documents available in \appNameBold{} documentation this one is of highest abstraction, which means that it only defines \textit{what} features will be available
    and not \textit{how} they will look or \textit{how} they will be implemented.

    \subsection{Intended Audience}
    Intended audience of this document are developers, designers and testers working on \appNameBold{}.

    \subsection{Scope}
    This document includes a summary of:
    \begin{itemize}
        \item how the system will function
        \item the scope of the project from the development viewpoint
        \item the technology used to develop the project
        \item the metrics used to determine the project's progress
        \item overall description
    \end{itemize}

    \section{Overall Description}
    Development teams need tools that let them distribute tasks among developers and track their progress.
    The problem is that because different development teams work with different methodologies they have to look for a tool that suits them.
    \appNameBold{} aims to become a go-to backlog application for development teams by providing high customizability and modularity.

    \subsection{Users}
    Target users are software development teams and individual developers.

    \subsection{Platform}
    The application will be developed in TypeScript with Bootstrap. It will connect to a REST API built with Java + Spring to store and retrieve data from a PostgreSQL database.

    \subsection{User Class and Characteristics}
    % TBD - not applicable in MVP

    \section{System Features and Requirement}
    \subsection{Functional Requirements}
    \subsection{External Interface Requirements}
    \subsection{System Features}
    \subsection{Non-functional Requirements}

\end{document}