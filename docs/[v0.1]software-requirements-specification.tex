% Preamble
\documentclass[a4paper]{article}

% Packages
\usepackage[utf8]{inputenc}
\usepackage{float}
\usepackage{caption}
\usepackage{tabularx}

% Settings
\captionsetup[table]{labelformat=empty}

% Commands
\newcommand{\appName}{BugTracker}
\newcommand{\appNameBold}{\textbf{BugTracker}}




% Document
\begin{document}

% Title page
    \begin{titlepage}
        \vspace*{\fill}
        \begin{center}
            \huge
            \appName{}
            \vspace{0.4cm}
            \Huge
            Software Requirement Specification
            \vspace{3cm}
            \begin{table}[h]
                \centering
                \caption{Document Revision History}
                \begin{tabular}{cccc}
                    \hline
                    Date       & Changed By    & Version & Comments               \\
                    \hline
                    2022-03-25 & Artur Kristof & 0.1     & Minimum viable product \\
                    \hfill     & \hfill        & \hfill  & \hfill                 \\
                    \hline
                \end{tabular}
            \end{table}
        \end{center}
        \vspace*{\fill}
    \end{titlepage}




    \tableofcontents




    \pagebreak
    \section{Introduction}
    This document details the project plan for the developments of \appNameBold{}.

    \subsection{Purpose}
    The purpose of this project is to create a web application that can be used as a backlog of tickets, bugs, tasks, user stories etc. by development teams working on building software.
    Out of all the documents available in \appNameBold{} documentation this one is of highest abstraction, which means that it only defines \textit{what} features will be available
    and not \textit{how} they will look or \textit{how} they will be implemented.

    \subsection{Intended Audience}
    Intended audience of this document are developers, designers and testers working on \appNameBold{}.

    \subsection{Scope}
    This document includes a summary of:
    \begin{itemize}
        \item how the system will function
        \item the scope of the project from the development viewpoint
        \item the technology used to develop the project
        \item the metrics used to determine the project's progress
        \item overall description
    \end{itemize}




    \pagebreak
    \section{Overall Description}
    Development teams need tools that let them distribute tasks among developers and track their progress.
    The problem is that because different development teams work with different methodologies they have to look for a tool that suits them.
    \appNameBold{} aims to become a go-to backlog application for development teams by providing high customizability and modularity.

    \subsection{Users}
    Target users are software development teams and individual developers.

    \subsection{Platform}
    The application will be developed in TypeScript with Bootstrap. It will connect to a REST API built with Java + Spring to store and retrieve data from a PostgreSQL database.

    \subsection{User Class and Characteristics}
% TBD - not applicable in v0.1




    \pagebreak
    \section{System Features and Requirements}

    \subsection{Functional Requirements}
    \begin{table}[htbp]
        \centering
        \begin{tabularx}{\linewidth}{|X|l|}
            \hline
            \centering{Description}                                                                   & \centering{Version} \tabularnewline \hline
            Users should be able to create new tickets with title and desription                      & 0.1                 \tabularnewline \hline
            Users should be able to edit title and description of a ticket                            & 0.1                 \tabularnewline \hline
            Users should be able to delete ticket                                                     & 0.1                 \tabularnewline \hline
            Users should be able to view list of tickets                                              & 0.1                 \tabularnewline \hline
            Users should be able to view separate tickets                                             & 0.1                 \tabularnewline \hline
            Users should be able to change ticket's status from "TO DO" to "DOING" and then to "DONE" & 0.1                 \tabularnewline \hline
        \end{tabularx}
    \end{table}

    \subsection{Non-functional Requirements}

    \subsubsection{Technologies Used}
    \begin{itemize}
        \item Java 17 OpenJDK
        \item Apache Maven
        \item Spring Boot
        \begin{itemize}
            \item Spring Web
            \item Spring Data JPA (Hibernate)
        \end{itemize}
        \item PostgreSQL 14
        \item HTML
        \item Bootstrap
        \item TypeScript
        \item WebPack
    \end{itemize}

    \subsubsection{Performance Requirements}
    \begin{itemize}
        \item The database should be normalized to prevent redundant data and improve performance
    \end{itemize}

    \subsubsection{Security Requirements}
    \begin{itemize}
        \item The database should be accessible only via the REST API
    \end{itemize}

    \subsubsection{Software Quality Attributes}
    \begin{itemize}
        \item Availability: % TBD - not applicable in current version
        \item Interoperability: % TBD - not applicable in current version
        \item Performance: Refer to Performance Requirements section in this document.
        \item Testability: The code must be unit tested with at least 75\% code coverage.
        \item Security: Refer to Security Requirements section in this document.
        \item Usability: The interface must be easy to learn without a tutorial and allow users to accomplish their goals without errors. The application must also accessible by people with disabilities.
        \item Functionality: The application must have every functionality required by Functional Requirements section in this document implemented.
    \end{itemize}
\end{document}
