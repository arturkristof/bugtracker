\documentclass[a4paper]{article}
\setcounter{tocdepth}{4}
\usepackage[utf8]{inputenc}
\usepackage{float}
\usepackage{caption}
\usepackage{tabularx}

\captionsetup[table]{labelformat=empty}
\newcommand{\appName}{BugTracker}
\newcommand{\appNameBold}{\textbf{BugTracker}}
\newcommand{\newLineParagraph}[1]{\paragraph{#1}\mbox{}\\}
\newcommand{\newLineSubParagraph}[1]{\subparagraph{#1}\mbox{}\\}

\begin{document}

    \begin{titlepage}
        \vspace*{\fill}
        \begin{center}
            \huge
            \appName{}

            \vspace{0.4cm}
            \Huge
            REST API Specification

            \vspace{3cm}
            \begin{table}[h]
                \centering
                \caption{Document Revision History}
                \begin{tabular}{cccc}
                    \hline
                    Date & Changed By & Version & Comments \\
                    \hline
                    2022-03-22 & Artur Kristof & 0.1 & \hfill \\
                    \hfill & \hfill & \hfill & \hfill \\
                    \hline
                \end{tabular}
            \end{table}
        \end{center}
        \vspace*{\fill}
    \end{titlepage}

    \tableofcontents

    \section{Introduction}
    This document details specification of \appNameBold{}'s REST API.

    \subsection{Intended Audience}
    Intended audience of this document are developers, designers and testers working on \appNameBold{}.

    \subsection{Scope}
    This document includes:
    \begin{itemize}
        \item \appNameBold{}'s REST API endpoints URIs
        \item available methods for each endpoint and their specifications
        \item possible responses
    \end{itemize}

    \section{Specification}

    \subsection{Assumptions}
    \begin{itemize}
        \item the API accepts JSON requests and produces JSON responses, therefore appropriate HTTP headers for every request are:
        \begin{verbatim}
            Content-Type: application/json
            Accept: application/json
        \end{verbatim}
        \item any unimplemented method must return \textbf{405 Method Not Allowed REQ-40501}
        \item if there are fields provided in the request that are not expected by method, \textbf{HTTP 400 Bad Request BOD-40004} must be returned
    \end{itemize}

    \subsection{v1}
    \subsubsection{/api/v1/tickets}
    \newLineParagraph{GET}
    % TODO: should support pagination in future versions
    Returns list of tickets with their ID and title. Does not accept request body. \\
    Supported HTTP response codes with examples:
    \subparagraph{HTTP 200 OK}
    \begin{verbatim}
        {
          "tickets": [
            {
              "id": 1,
              "title": "ticket1"
            },
            {
              "id": 2,
              "title": "ticket2"
            }
          ]
        }
    \end{verbatim}
    Returned when the request is correct (correct HTTP headers and empty body) and tickets have been found on the server. If there are no tickets return an empty list:
    \begin{verbatim}
        {
          "tickets": []
        }
    \end{verbatim}

    \newLineSubParagraph{HTTP 400 Bad Request}
    \begin{itemize}
        \item \textbf{BOD-40001} when body is not empty
    \end{itemize}

    \newLineParagraph{POST}
    Used for creating new tickets. Accepts body according to following example:
    \begin{verbatim}
        {
          "title": "abc", // required, non-null, non-empty
          "description": "", // not required, default ""
          "statusCode": "" // not required, default "NEW"
        }
    \end{verbatim}
    Supported HTTP response codes with examples:
    \subparagraph{HTTP 201 Created}
    \begin{verbatim}
        {
          "id": "1",
          "title": "abc",
          "description": "",
          "statusCode": "NEW"
        }
    \end{verbatim}
    Returns whole ticket object when created successfully.

    \newLineSubParagraph{HTTP 400 Bad Request}
    \begin{itemize}
        \item \textbf{BOD-40002} when body is not provided or is empty
        \item \textbf{BOD-40003} when title is not provided or is empty.
    \end{itemize}

    \newLineSubParagraph{HTTP 422 Unprocessable Entity}
    \begin{itemize}
        \item \textbf{DAT-42201} when provided status doesn't exist
    \end{itemize}

    \subsubsection{/api/v1/tickets/:id}
    \newLineParagraph{GET}
    Returns details about ticket with given ID. Does not accept request body. \\
    Supported HTTP response codes with examples:
    \subparagraph{HTTP 200 OK}
    \begin{verbatim}
        {
          "id": "1",
          "title": "abc",
          "description": "",
          "statusCode": "NEW"
        }
    \end{verbatim}
    \newLineSubParagraph{HTTP 400 Bad Request}
    \begin{itemize}
        \item \textbf{BOD-40001} when body is not empty
    \end{itemize}
    \newLineSubParagraph{HTTP 404 Not Found}
    \begin{itemize}
        \item \textbf{URI-40401} when ticket with provided ID does not exist
    \end{itemize}

    \newLineParagraph{PUT}
    Used for editing ticket with given ID. Overrides the ticket with provided data and deletes of sets to default data that is not provided. Accepts body according to following example:
    \begin{verbatim}
        {
          "title": "abc", // required, non-null, non-empty
          "description": "", // not required
          "statusCode": "" // not required
        }
    \end{verbatim}
    Supported HTTP response codes with examples:
    \subparagraph{HTTP 200 OK}
    \begin{verbatim}
        {
          "id": "1",
          "title": "abc",
          "description": "",
          "statusCode": "NEW"
        }
    \end{verbatim}
    Returns whole ticket object when updated successfully.
    \newLineSubParagraph{HTTP 400 Bad Request}
    \begin{itemize}
        \item \textbf{BOD-40002} when body is empty
        \item \textbf{BOD-40003} when title is not provided or is empty.
    \end{itemize}
    \newLineSubParagraph{HTTP 404 Not Found}
    \begin{itemize}
        \item \textbf{URI-40401} when ticket with provided ID does not exist
    \end{itemize}
    \newLineSubParagraph{HTTP 422 Unprocessable Entity}
    \begin{itemize}
        \item \textbf{DAT-42201} when provided status doesn't exist
    \end{itemize}


    \newLineParagraph{PATCH}
    Used for editing ticket with given ID. Only updates provided data and leaves everything else untouched. Accepts body according to following example:
    \begin{verbatim}
        {
          "title": "abc", // not required, non-null, non-empty
          "description": "", // not required
          "statusCode": "" // not required
        }
    \end{verbatim}
    Supported HTTP response codes with examples:
    \subparagraph{HTTP 200 OK}
    \begin{verbatim}
        {
          "id": "1",
          "title": "abc",
          "description": "",
          "statusCode": "NEW"
        }
    \end{verbatim}
    Returns whole ticket object when updated successfully.
    \newLineSubParagraph{HTTP 400 Bad Request}
    \begin{itemize}
        \item \textbf{BOD-40002} when body is empty
        \item \textbf{BOD-40005} when title is provided but is empty or null.
    \end{itemize}
    \newLineSubParagraph{HTTP 404 Not Found}
    \begin{itemize}
        \item \textbf{URI-40401} when ticket with provided ID does not exist
    \end{itemize}
    \newLineSubParagraph{HTTP 422 Unprocessable Entity}
    \begin{itemize}
        \item \textbf{DAT-42201} when provided status doesn't exist
    \end{itemize}

    \newLineParagraph{DELETE}
    Used for deleting ticket with given ID. Does not accept body. \\
    Supported HTTP response codes:
    \subparagraph{HTTP 204 No Content}
    \newLineSubParagraph{HTTP 400 Bad Request}
    \begin{itemize}
        \item \textbf{BOD-40001} when body is not empty
    \end{itemize}
    \newLineSubParagraph{HTTP 404 Not Found}
    \begin{itemize}
        \item \textbf{URI-40401} when ticket with provided ID does not exist
    \end{itemize}

    \subsubsection{Error responses}
    Error responses are standarized throughout the API and their bodies must be built according to following template:
    \begin{verbatim}
        {
          "errorCode": "",
          "message": "",
          "detail": ""
        }
    \end{verbatim}
    Below you can find error specification:
    \begin{table}[htbp]
        \centering
        \caption{Error Response Specification}
        \begin{tabularx}{\linewidth}{|l|X|X|}
            \hline
            \centering{Error Code} & \centering{Message} & \centering{Detail} \tabularnewline \hline
            BOD-40001 & Request's body is incorrect. & Expected empty body, got request with non-empty body. \tabularnewline \hline
            BOD-40002 & Request's body is incorrect. & Expected body, got request with empty or no body. \tabularnewline \hline
            BOD-40003 & Request's body is incorrect. & Expected title, got request with empty or no title. \tabularnewline \hline
            BOD-40004 & Request's body is incorrect. & There are fields provided that are not expected by the method called. \tabularnewline \hline
            BOD-40005 & Request's body is incorrect. & Title is not required but provided title must not be empty or null. \tabularnewline \hline
            DAT-42201 & Request's data is incorrect. & Provided ticket status doesn't exist. \tabularnewline \hline
            URI-40401 & Not found. & Ticket with provided ID does not exist. \tabularnewline \hline
            REQ-40501 & Request is incorrect. & Method you are requesting has not been implemented. \tabularnewline \hline
            \hfill & \hfill & \hfill \tabularnewline \hline
        \end{tabularx}
    \end{table}

\end{document}