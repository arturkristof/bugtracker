\documentclass[a4paper]{article}
\usepackage[utf8]{inputenc}
\usepackage{float}
\usepackage{caption}

\captionsetup[table]{labelformat=empty}
\newcommand{\appName}{BugTracker}
\newcommand{\appNameBold}{\textbf{BugTracker}}
\newcommand{\newLineParagraph}[1]{\paragraph{#1}\mbox{}\\}

\begin{document}

    \begin{titlepage}
        \vspace*{\fill}
        \begin{center}
            \huge
            \appName{}

            \vspace{0.4cm}
            \Huge
            REST API Specification

            \vspace{3cm}
            \begin{table}[h]
                \centering
                \caption{Document Revision History}
                \begin{tabular}{cccc}
                    \hline
                    Date & Changed By & Version & Comments \\
                    \hline
                    2022-03-22 & Artur Kristof & 0.1 & \hfill \\
                    \hfill & \hfill & \hfill & \hfill \\
                    \hline
                \end{tabular}
            \end{table}
        \end{center}
        \vspace*{\fill}
    \end{titlepage}

    \tableofcontents

    \section{Introduction}
    This document details specification of \appNameBold{}'s REST API.

    \subsection{Intended Audience}
    Intended audience of this document are developers, designers and testers working on \appNameBold{}.

    \subsection{Scope}
    This document includes:
    \begin{itemize}
        \item \appNameBold{}'s REST API endpoints URIs
        \item available methods for each endpoint and their specifications
        \item possible responses
    \end{itemize}

    \section{Specification}

    \subsection{Assumptions}
    \begin{itemize}
        \item the API accepts JSON requests and produces JSON responses, therefore appropriate HTTP headers for every request are:
        \begin{verbatim}
            Content-Type: application/json
            Accept: application/json
        \end{verbatim}
    \end{itemize}

    \subsection{v1}
    \subsubsection{/api/v1/tickets}
    \newLineParagraph{GET}
    % TODO: should support pagination in future versions
    Returns list of tickets with their ID and name. Returns every ticket so there is no need for any request body. \\
    Supported HTTP response codes with examples:
    \subparagraph{HTTP 200 OK}
    \begin{verbatim}
        {
          "tickets": [
            {
              "id": 1,
              "name": "ticket1"
            },
            {
              "id": 2,
              "name": "ticket2"
            }
          ]
        }
    \end{verbatim}
    Returned when the request is correct (correct HTTP headers and empty body) and tickets have been found on the server.

    \newLineParagraph{POST}

\end{document}